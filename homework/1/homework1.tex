\documentclass{article}

\usepackage{fullpage}
\usepackage{hyperref}
\hypersetup{
  colorlinks = true
}
\usepackage{listings}
\lstset{basicstyle=\ttfamily\footnotesize,breaklines=true}


\title{Homework 1---ECE590--001---{\bf Due: 29 Jan 2020}}
\date{1/15/2020}
\begin{document}
\maketitle
\begin{enumerate}
\item
  Install Docker on your local machine. Docker is available for Windows, Mac,
and Linux. You have succeeded when the following works.
\item
  Run your first Docker commands:
  \begin{enumerate}
  \item 
    \begin{lstlisting}[language=bash]
      docker pull continuumio/anaconda3 # pull the docker image from the container registry
      docker run -i -t continuumio/anaconda3 /bin/bash # run the image interactively
      # you will not have a root prompt within an anaconda3 containers
      exit # exits the terminal session (/bin/bash) and closes terminates the container
    \end{lstlisting}
  \item Read more \url{https://hub.docker.com/r/continuumio/anaconda3}. {\em try the last command
      to get an interactive jupyter notebook bridged from docker to your local machine}
  \end{enumerate}
\item
  Install \href{https://github.com/openai/spinningup}{spinningup} and its
dependencies by docker container and issuing commands. It is installed
correctly when you can run the algorithms provided.
\begin{enumerate}
\item Start by experimenting:
  \begin{lstlisting}[language=bash]
    docker run -i -t continuumio/anaconda3 /bin/bash
    git clone https://github.com/openai/spinningup # !!! these changes are not persistent
    pip install spinningup # use pip on local copy to install spinning up
  \end{lstlisting}
\item The dependencies will not be met by the {\tt pip} command above. But you can pip these
  requirements into the base anaconda environment. The requirements can be found:
  \url{https://github.com/openai/spinningup/blob/master/setup.py} and the command would be something
  like
  \begin{lstlisting}[language=bash]
    pip install ``req1'' ... ``reqN''
  \end{lstlisting}
\item The last pip command will not succeed (i.e. run to completion without errors). Read these errors and try to fix them.
  {\em The errors are due the fact that you have minimal development environment within the Docker container, you need to install more software and libraries,
    in particular, you likely need c/c++ compilers to support certain requirements that were within setup.py}. Figure out
  what these are and write them down. Additionally, how did you install them? {\em Hint: {\tt apt install} is the package management system in Debian and can install
    most things.}
  \begin{lstlisting}[language=bash]
    apt install emacs # installs emacs text editor
    apt install g++ gcc # installs gnu c++ and c compilers
  \end{lstlisting}
\item Run something from Spinning Up:
  \begin{lstlisting}[language=bash]
    python spinup/examples/bench_ppo_cartpole.py
  \end{lstlisting}
\end{enumerate}
\item
  From github (you will need to establish a github account), Fork {\tt
ece590hineman} to a copy of the repository in your own name space.
Clone a copy of the repo from your name space (you can version control your work
this way). {\bf ((edit)): You may also use \url{gitlab.com}. Essentially, what I need is way
  for you to share code and version control it.}
\item
  So far we have executed docker from an image and editted it, we would now like
to build an image locally from a dockerfile that includes all the dependencies
we have for spinning up in persistent and repeatable way. There is already a
directory in {\tt ece590hineman/homework/1/solutions} that contains a
skeleton dockerfile. Edit this file to install spinning up and its dependencies.
Show that you've succeed by screening shotting
\begin{lstlisting}[language=bash]
  python spinup/examples/bench_ppo_cartpole.py
\end{lstlisting}
You'll use what you learned from the last exercise to complete this.
\end{enumerate}
\end{document}
